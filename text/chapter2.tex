\chapter{Описание предлагаемого метода} 
\label{chapter2}

	Однонуклеотидный полиформизм (ОНП) является наиболее популярным видом мутации. В диплойдном организме различаются два типа ОНП. Гомозиготный ОНП отличается от референсного генома в обеих аллелях, в то время как гетерезиготный ОНП только в одной аллеле. Зависимая от контекста ошибка может быть принята за гетерозиготный ОНП, когда некоторая часть чтений отличается от референса. Для того, чтобы уменьшить количество таких случаев, был разработан предлагаемый метод.

\section {Основная идея метода}

	Основная идея заключается в следующем. Так как молекула ДНК состоит из двух цепочек, секвенатор может получать чтения с любой из них. Поэтому чтение, которое мапится на прямой геном, имеет прямое направление, а которое мапится на обратно комплементарный геном, соответственно, имеет обратное направление. В итоге получается, что на некоторую геномную позицию замапились чтения, идущие в противоположных направлениях. Поэтому нуклеотидам, замапленных на одинаковую геномную позицию, в противоположных чтениях предшествуют разные контексты. Тогда в ситуации, когда некоторый контекст склонен к ошибке, в чтениях с противоположной стороны находится другой контекст, и ошибок не возникает. Вероятность наличия плохих контекстов в одной геномной позиции с двух сторон крайне мала, и такие случаи опускаются. Ошибка после контекста не зависит от позиции и возникает с некоторой вероятностью всегда во всех позициях, где этот контекст появляется. Поэтому, было бы правильным, собрать статистику для всех контекстов по всем их позициям, и оценить насколько больше ошибок возникает с одной стороны, чем с другой. Стоит заметить, что при наличии в позиции мутации, ошибки возникают с обеих сторон в примерно одинаковом количестве, и они не будут влиять на общую картину.

	Предлагаемый метод можно разбить на следующие части:
	\begin{enumerate}
	\item построение таблицы сопряжённости для каждого геномного мотива;
	\item отбор наиболее склонных к ошибкам мотивов на основе построенных таблиц;
	\item выбор наиболее однозначных (в плане настоящего генотипа) позиций в геноме;
	\item подсчет статистики для мотивов чтений по отобранным позициям.
	\end{enumerate}

\section {Построение таблицы сопряжённости}

	Для данного мотива $m$ определяются все вхождения его в референсный и обратно комплементарный геном. Будем называть геномный интервал, где мотив совпадает с прямым референсом, $F$-интервалом и последнюю позицию $F$-интервала $F$-позицией. Также интервал, где мотив совпадает с обратно комплементарным референсом, $R$-интервалом и его первую позицию $R$-позицией. Геномная позиция $i$ всегда ссылается на прямой референс. Будем называть чтение, которое замапилось на прямой референс, $F$-чтением, на обратное --- $R$-чтением.
	
	Совпадение произошло в позиции $i$, если чтение является $F$-чтением и нуклеотид в прямом референсе в позиции $i$ совпал с нуклеотидом в чтении, или если чтение является $R$-чтением и нуклеотид в прямом референсе в позиции $i$ совпал с комплементарным к соответствующему нуклеотиду в чтении. В других случаях происходит несовпадение. По конвенции нуклеотиды в $R$-чтениях уже комплементированы. Поэтому пайлап можо всегда сравнивать с прямым референсом.  В одной $R$-позиции на рис. ~\ref{table_construction}, в то время как пайлап показывает $A \to G$ несовпадение во многих $R$-чтениях, это по факту технически $T \to C$ несовпадение по этой конвенции. Все три позиции, показанные на этом гипотетическом примере, постоянно показывали бы смещение ошибок к одному направлению, то есть контекстно-зависимую ошибку (более точнее, $T \to C$ несовпадение после $CCAGAC$). Таблица сопряжённости для мотивов вычисляется следующим образом:
	\begin{enumerate} 
	\item инициализировать $a=b=c=d=0$;
 	\item для каждой $F$-позиции мотива m получаем пайлап и увеличиваем $а$ на число совпадений в $F$-чтениях, $b$ на число несовпадений в $F$-чтениях, $с$ на число совпадений в $R$-чтениях и $d$ на число несовпадений в $R$-чтениях;
	\item для каждой $R$-позиции мотива $m$ получаем пайлап и увеличиваем $а$ на число совпадений в $R$-чтениях, $b$ на число несовпадений в $R$-чтениях, $с$ на число совпадений в $F$-чтениях и $d$ на число несовпадений в $F$-чтениях.
	\end{enumerate} 
	
\begin{figure}[h!]
	\center{\includegraphics[width=1.0\textwidth]{table_creation}}
	\caption{Построение таблицы для мотива CCAGACT. Прямой референс (5' --- 3') изображен сверху, ниже комплементарный ему (3' --- 5'). $F$-чтения обозначены красными стрелками, $R$-чтения --- синими. $F$-интервал помечен на прямом референсе, $R$-интервал на обратно комплементарном. Две $F$-позиции и одна $R$-позиция отмечены вертикальными столбцами. Таблица для каждой такой позиции и суммарная итоговая для мотива $CCAGACT$ показыны после чтений внизу.}
	\label{table_construction}
	\end{figure}

Для начала нужно подсчитать число совпадений и различий с геномом в каждом из двух направлений для каждой геномной позиции в замапленных чтениях. Псевдокод представлен в листинге ~\ref{calcGenomer}.

           \begin{algorithm}[h!]
	\caption{Подсчет статистики по геномным позициям}
	\label{calcGenomer}
	%{\fontsize{12}{12}\selectfont
	\begin{algorithmic}[1]
	\FOR {всех чтений в наборе}
		\FOR {каждой позиции в чтении}
			\STATE $pos$ --- {соответствующая геномная позиция}
			\STATE $ref$ ---{ референсный нуклеотид}
 			\STATE $nucl$ --- {текущий нуклеотид в чтении}
			\IF {(чтение обратное)}
				\IF {($nucl$ равен $ref$)}
					\STATE $rm[pos]$ {добавить единицу}
				\ELSE
					\STATE $rmm[pos]$ {добавить единицу}
			           \ENDIF
                                \ELSE 
				\IF {($nucl$ равен $ref$)}
                                		\STATE $fm[pos]$ {добавить единицу}
				\ELSE
					\STATE $fmm[pos]$ {добавить единицу}
				\ENDIF
			\ENDIF
		\ENDFOR
	\ENDFOR
           \STATE $stat$ = $(fm, rm, fmm, rmm)$ 
	\end{algorithmic}
	%}
	\end{algorithm}

После этого этапа известно состояние покрытия каждой геномной позиции. Нас интересуют только хорошо и равномерно покрытые позиции. Дело в том, что если в некоторой геномной позиции случится сильно смещенное в одну сторону покрытие и одновременно в ней же мутация, алгоритм неверно расценит такую ситуацию в сторону наличия плохого контекста. Поэтому отбираются позиции, чьи покрытия соотвествуют следующим условиям:
	\begin{itemize}
		\item покрытие с каждой из стороной должно быть больше или равно $3$;
		\item отношение покрытия с большей стороны к меньше стороне не должно превосходить $5$.
	\end{itemize}

Далее  в листинге \ref{calcQmer} приведен псевдокод алгоритма подсчета таблицы сопряжённости для каждого $k$-мера.

 \begin{algorithm}[h!]
	\caption{Подсчет статистики для к-мера с помощью статистики для геномных позиций}
	\label{calcQmer}
	%{\fontsize{12}{12}\selectfont
	\begin{algorithmic}[1]
	\REQUIRE
		$k$ --- размер к-мера
		$map$ --- хэш-таблица для каждого к-мера
		$stat$ --- статистика покрытия для геномной позиции
	\FOR {всех K-мер, встречающихся в геноме}
		\FOR {каждой позиции вхождения K-мера в геном}
			\STATE $first$ --- {позиция первого элемента К-мера}
			\STATE $last$ --- {позиция последнего элемента}
			\STATE {Добавить статистику последней позиции для к-мера} $map[kmer] += stat[last]$
			\STATE {Добавить статистику первой позиции для обратно комплементарного к-мера} $map[rev\_comp(kmer)] += stat[first]$
		\ENDFOR
	\ENDFOR 
	\end{algorithmic}
	%}
	\end{algorithm}

Теперь имеется суммарная статистика для всех вхождений каждого к-мера, встречающегося в геноме. Заметим, что если существует ошибка, индуцируемая некоторым контекстом, то в чтении после это контекста ошибки будут появлятся чаще, чем в чтениях, идущих в противоположном направлении. Нужно лишь посчитать различие в вероятности возникновения ошибки с разных сторон. И чем она больше, тем хуже контекст. 

\subsection{Точный тест Фишера}

Точный тест Фишера --- тест статистической значимости взаимосвязи между двумя переменными в таблице сопряженности признаков размерности $2 \times 2$.

Точный тест Фишера вычисляет $p$-значение из таблицы сопряжённости $2 \times 2$ для определенения независимости двух характеристик данных: <<направление чтения>> и <<доля несовпадений>>, что эквивалентно проверке, что строки имеют одинакое распределение. Если это так, то в данном случае нет основания на наличие смещения ошибок по направлениям. 

\begin{table}[!h]
\begin{center}
\resizebox{0.8\textwidth}{!}{
\begin{tabular}{ M{2.5cm} M{2.5cm} M{2.5cm} M{1.5cm} }
\hline
 & \textbf{Совпадение} & \textbf{Несовпадение} & \textbf{Всего} \\ \hline
Прямое & a & b & f \\ \hline
Обратное & c & d & k\\ \hline
Всего & m & s & n\\ \hline
\end{tabular}}
\end{center}
\captionsetup{justification=centering}
\caption{Таблица сопряженности $2 \times 2$.}
\label{contingency_table}
\end{table} 

Для расчета $p$-значения таблицы ~\ref{contingency_table}, тест Фишера предполагает, что все маргинальные итоги заданы и фиксированы. Обозначим маргинальную информацию как $\mathcal{M}\ =\ (f,\ k,\ m,\ s,\ n)$, где $n\ =\ f\ +\ k\ =\ m\ +\ s$. Дано $\mathcal{M}$ и один элемент таблицы (без потери общности пусть это $a$), можно вычислить все остальные элементы, и как раз $(a|\mathcal{M})$ обозначает такое представление таблицы. Вероятность нулевой гипотезы $Pr_{H_0}(a|\mathcal{M})$ наблюдаемой таблицы $(a|\mathcal{M})$:

$$Pr_{H_0}(a|\mathcal{M}) = \frac {\begin{pmatrix} a + b \\ a \end{pmatrix} \cdot \begin{pmatrix} c + d \\ c \end{pmatrix}} {\begin{pmatrix} a + b + c + d \\ a + c \end{pmatrix}}$$

$P$-значение $(a|\mathcal{M})$ --- это вероятность наблюдать такую или или более экстремальную таблицу при нулевой гипотезе. Под более экстремальной таблицей подразумевается таблица с меньшей вероятностью, чем данная:

$$ p{-}value (a|\mathcal{M}) =  \sum_{a^{'} \in E(a, \mathcal{M})} Pr_{H_0}(a^{'}|\mathcal{M}),$$ где <<экстремальные>> значения $a^{'}$ из множества $E(a, \mathcal{M}) = \{a^{'}: Pr_{H_0}(a^{'}|\mathcal{M}) \le Pr_{H_0}(a|\mathcal{M})\}$.

Если p-значение достаточно мало, нулевая гипотеза отклоняется, подразумевая что две строки не были получены из одного распределения. Величину $-log_{10}(p{-}value(a|\mathcal{M}))$ можно рассматривать как количественную меру смещения.

Тест Фишера вычислительно сложен для таблиц с большими значениями и может быть заменен на ${\chi}^2$-тест в этом случае [13].

\subsection{Множественное тестирование}

При проведении многих статистических тестов ожидаемое количество ложноположительных результатов может быть тоже большим. Существует много стратегий борьбы с такими ситуациями в множественной проверки гипотез. Например, одним из популярных подходов является контроль групповой вероятности ошибки (первого рода) в виде поправки Бонферрони.

Пусть $H_1,..., H_m$ --- набор гипотез, а $p_1, ..., p_m$ --- их p-значения, $I_0$ --- неизвестное подмножество $I$ истинных нулевых гипотез. Групповая вероятность ошибки ($family{-}wise\ error\ rate$, $FWER$) --- вероятность отклонения как минимум одной гипотезы из $I_0$ (получения одной ошибки первого рода как минимум).  
Метод Бонферрони говорит, что для достижения $FWER \le \alpha$, достаточно отвергать гипотезы с $p_i < \frac {\alpha} {m}$, где $m$ --- количество всех гипотез.

В нашем случае p-значения важных мотивов очень малы по причине большого количества данных (чтений) и, отчасти, нашей стратегии объединения по контекстам. Таким образом, мы можем выбрать метод поправки Бонферрони для уменьшения потери многих важных мотивов. В таком случае поправка Бонферрони будет равна $T = \frac {\alpha} {|S(q)|}$, где $\alpha$ --- константа, задающая желаемый уровень ошибок, а $|S(q)| = 4^q$ --- мощность множества мотивов длины $q$, каждый элемент которых принадлежит алфавиту ${\Sigma} = \{A,\ C,\ G,\ T\}$.   

\subsection{Отбор существенных мотивов}

Поскольку нулевой гипотезой являлось независимость направления и числа ошибок по направлениям, нужно наоборот оставить те мотивы, где присутствует смещение, то есть чьи уровни значимости меньше или равны порогу Бонферрони: $p_i \le T$. 

Для всех таких мотивов подсчитывается прямая вероятность ошибки $FER = \frac {b} {a+b}$ и обратная вероятность ошибки $RER= \frac {d} {c+d}$ в обозначениях таблицы. Потом удаляются мотивы, имеющие $RER \ge \epsilon$, то есть мотивы, при которых вероятности ошибки с обеих сторон велики. Для оставшихся мотивов подсчитывается различие в вероятностях ошибок с каждой стороны $ERD = FER - RER$. Удаляются мотивы со слишком маленьким $ERD$, то есть $ERD < \delta$.

После определения наиболее склонных к ошибкам контекстов нужно оценить эту самую вероятность ошибки после них. По причине того, что в настоящем геноме присутствуют естественные мутации, нужно отобрать наиболее уверенные позиции, где не возникает противоречний мутация там или нет.  
Для этого по принципу максимального правдоподобия осуществляется отбор.

\section {Отбор по принципу консенсуса}

По принципу консенсуса, если в данной позиции только один тип нуклеотида, то он и будет выбран. Если два или больше, обычно обращают внимание только на два наиболее часто встречающихся. Можно считать, что позиция покрыта только двумя типами нуклеотидов. Этого можно добится, если расценивать другие типы как ошибки. 

Пусть позиция покрыта $n$ чтениями. Для нуклеотида $i$-го чтения настоящий нуклеотид $B_i$ и наблюдаемый $\hat{B_i} = \hat{b}^{(i)}$ с вероятностью ошибки ${\epsilon}_i$. Для определенности будем считать первые $k$ нуклеотидов как $b_1$ (то есть $\hat{b}^{(1)} = ... = \hat{b}^{(k)} = b_1$) и оставшиеся $n-k$ как $b_2$ (то есть $\hat{b}^{(k+1)} = ... = \hat{b}^{(n)} = b_2$). Имеем:
\begin{eqnarray*}
P(D|\langle b_2, b_2 \rangle) = Pr\{\hat{B}_1 = ... = \hat{B}_k = b_1, \hat{B}_{k+1} = ... = \hat{B}_n = b_2|\langle b_2, b_2 \rangle  \} \\
= \prod_{i=1}^{k}{\epsilon}_i \cdot \prod_{j=k+1}^{n}(1-{\epsilon}_j)
\end{eqnarray*}
$$P(D|\langle b_1, b_1 \rangle) = \prod_{i=1}^{k}{(1 - \epsilon}_i) \cdot \prod_{j=k+1}^{n}{\epsilon}_j$$
и
\begin{align*}
&P(D|\langle b_1, b_2 \rangle) \\
=\ &Pr\{\hat{B}_1 = ... = \hat{B}_k = b_1, \hat{B}_{k+1} = ... = \hat{B}_n = b_2|\langle b_1, b_2 \rangle  \} \\
=\ & \sum_{a_1=1}^2 ... \sum_{a_n=1}^2 Pr \{ \hat{B}_1 = ... = \hat{B}_k = b_1, \hat{B}_{k+1} = ... = \hat{B}_n = b_2|B_1=b_{a_1}, ..., B_n=b_{a_n} \} \\
&\cdot Pr \{B_1 = b_{a_1}, B_2 = b_{a_2}, ..., B_n = b_{a_n}|\langle b_1, b_2 \rangle\} \\
=\ &\frac{1}{2^n} \prod_{i=1}^{n} \sum_{a_i = 1}^{2} Pr \{\hat{B}_i = \hat{b}^{(i)} | B_i = b_{a_i}\}
\end{align*}

Теперь по формуле Байеса:
$$P(\langle b_1, b_2 \rangle | D) = \frac {P(D|\langle b_1, b_2 \rangle) P(\langle b_1, b_2 \rangle)} {P(D)}$$

По принципу максимального правдоподобия предполагаемым генотипом будет тот, который максимизирует эту апостериорную вероятность:
$$\hat{g} = argmax\ P(g | D) = argmax\ P(D|g)\ P(g)$$

Априорная вероятность того, что генотип является гетерозиготным $g=\langle b, b' \rangle$, равняется $r = 10^{-3}$. Тогда вероятность увидеть гомозиготный генотип $P(\langle b, b \rangle)  = 1 - 2r$, так как генотип не упорядочен. 

Для диплойдного организма в геномной позиции может быть один из десяти возможных генотипов: четыре гомозиготных ($AA$, $CC$, $GG$, $TT$) и шесть гетерозиготных ($AC$, $AG$, $AT$, $CG$, $CT$, $GT$). Из всех этих генотипов выбираются три наиболее вероятных, чьи нуклеотида являются двумя доминантными. Например, если в некоторой позиции встречаются $A$ и $C$ чаще других, тогда возможные генотипы будут $AA$, $CC$, $AC$.

В силу того, что показатели качества у нуклеотидов являются не точными и хочется больше гарантий в правильности выбора, нужно сделать <<зазор>> между наиболее правдоподобным генотипом и вторым после него. Зазор представляет собой отношение логарифмов соответствующих вероятностей, которое не должно быть меньше некоторого <<доверительного>> значения. В том случае, если два наиболее правдоподобных генотипа находятся <<близко>> друг другу, нельзя придти к консенсусу в такой позиции. Позиции с правдоподобным гетерозиготным генотипом нужно исключать из доверительных, поскольку в них велика вероятность мутации, чем ошибки.

В итоге отбираются позиции удовлетворяющие следующим условиям:
	\begin{itemize}
		\item покрытие с каждой из стороной должно быть больше или равно 3;
		\item отношение покрытия с большей стороны к меньшей стороне не должно превосходить 5;
		\item с каждой из сторон достигнут консенсус в выборе генотипа, эти генотипы совпадают и являются гомозиготными. В том случае, если с одной из сторон присутствует плохой контекст, такая сторона исключается, и генотип оставшейся стороны должен быть гомозиготным. В случае, если с двух сторон присутствуют плохие контексты, то такая позиция исключается;
	\end{itemize}

\section{Анализ контекстов чтений}

По оставшимся <<уверенным>> позициям собирается статистика по теперь уже контекстам чтений.

 	\begin{algorithm}[h!]
	\caption{Подсчет статистики для к-мера с помощью статистики для геномных позиций}
	\label{calcReadQmer}
	%{\fontsize{12}{12}\selectfont
	\begin{algorithmic}[1]
	\REQUIRE
		$k$ --- размер к-мера
		$map$ --- хэш-таблица для каждого к-мера и качества его последнего элемента
	\FOR {всех чтений в наборе}
		\FOR {всех K-мер, встречающихся в чтении}
			\STATE $first\_phred$ --- {качество первого элемента К-мера}
			\STATE $last\_phred$ --- {качество последнего элемента}
			\STATE {$ref$ --- референсный нуклеотид}
 			\IF {(чтение обратное)}
				\STATE {$map[\langle reverse\_complement(kmer), first\_phred \rangle][ref]$ добавить единицу}
			\ELSE 
				\STATE {$map[\langle kmer, last\_phred \rangle][ref]$ добавить единицу}
			\ENDIF
		\ENDFOR
	\ENDFOR 
	\end{algorithmic}
	%}
	\end{algorithm}

Теперь для каждого $k$-мера и показателя качества его последнего элемента имеется число совпадений с каждым типом нуклеотида референсного генома, который соответствовал этому последнему элементу в чтении. В идеале все совпадения должны приходиться на этот самый последний нуклеотид (например, для всех $ACCT$ в чтениях для нуклеотида $T$ в соответствующей позиции в референсе тоже должен быть $T$, в идеале всегда). Но на практике в реальных данных присутствуют помехи --- ошибки секвенирования. Поэтому секвенатор присваивает показатель своей уверенности к каждому нуклеотиду. Опять же, в идеале, этот показатель уверенности должен соответствовать числу ошибок, которые были получены статистикой. На деле, даже при подсчете по <<уверенным>> позициям это не так. Это получается по нескольким причинам: 
\begin{itemize}
	\item качество, данное секвенатором, не соответствует действительности;
	\item в полученных <<уверенных>> позициях присутствуют трудноразличимые мутации (даже специальными алгоритмами для их поиска);
	\item наличие контекстно-зависимых ошибок секвенирования.
\end{itemize}
Но мутаций ограниченное число, и они случаются независимо, то есть вносят ошибки во все контексты примерно одинаково, когда как контекстные ошибки влияют только на определенные контексты и значительно портят их статистику совпадений. Именно по этой статистике и определяются наиболее значительные ошибки и их контексты. После всего останется только заменить в наборе чтений показатели качества этих контекстов на соответствующие.

\section{Выводы по главе \protect\ref{chapter2}}

Был описан предлагаемый метод для вычисления вероятностей ошибок на основе последовательности нуклеотидов перед ними в чтениях. Описаны возникающие проблемы и способы борьбы с ними.
