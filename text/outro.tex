\startconclusionpage

Современные секвенаторы совершают ошибки в определении правильного нуклеотида и присновенные ими показатели качества не соответствуют действительности. В настоящей работе был предложен метод определения вероятности ошибки секвенирования в зависимости от её контекста. Был реализован соответствующий алгоритм и опробован на человеческом геноме. В ходе эксперимента метод позволил повысить эффективность алгоритма поиска мутаций путём замены показателей качества нуклеотидов в чтениях на рассчитанные на основе их контекста.
