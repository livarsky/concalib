\chapter{Результаты работы метода} 
\label{chapter3}

	Предложенный метод был опробован на чтениях генома человека. За референсный геном была взята сборка $GRCh37$ ($hg19$, \emph{Genome Reference Consortium Human Reference 37} ($GCA\_000001405.1$) []. В качестве чтений была взяла библиотека парных чтений $SRR622461$ [], предоставленная $1000\ Genomes$ [].
Чистые чтения были замаплены с помощью программы BWA (Burrows-Wheeler Aligner []) командами:
\begin{align*}
& bwa\ index\ -a\ bwtsw\ ucsc.hg19.fasta \\
& bwa\ aln\ -q\ 15\ -f\ read1.sai\ ucsc.hg19.fasta\ SRR622461\_1.filt.fastq.gz \\
& bwa\ aln\ -q\ 15\ -f\ read2.sa\i ucsc.hg19.fasta\ SRR622461\_2.filt.fastq.gz \\
& bwa\ sampe\ -a\ 750\  -f\ fullgenome.sam\ ucsc.hg19.fasta\ read1.sai\ read2.sai\ \\ 
& SRR622461\_1.filt.fastq.gz\ SRR622461\_2.filt.fastq.gz
\end{align*}

Далее были удалены дубликаты с помощью $Picard\ MarkDuplicates$[] и были удалены неуникальные чтения, которые содержатся в разных позициях, с помощью $SamTools$[]. В таблице ~\ref{tab:cont} показано число чтений, замапленных на каждую хромосому, и среднее покрытие каждой геномной позиции.

\begin{table}[H]
\begin{center}
\begin{tabular}{|c|c|c|c|}
\hline
Номер хромосомы & Длина & Количество соотв. чтений  & Ср. покрытие \\
\hline
1 & 249250621 &  11624327 & 4.66\\
\hline
2 & 243199373 &  12297590 & 5.06\\
\hline
3 & 198022430 &  10139425 & 5.12\\
\hline
4 & 191154276 &  9819297 & 5.14\\
\hline
5 & 180915260 &  9153074 & 5.06\\
\hline
6 & 171115067 &  8584662 & 5.02\\
\hline
7 & 159138663 &  7946173 & 4.99\\
\hline
8 & 146364022 &  7336225 & 5.01\\
\hline
9 & 141213431 &  5780517 & 4.09\\
\hline
10 & 135534747 &  6904404 & 5.09\\
\hline
11 & 135006516 &  6677618 & 4.94\\
\hline
12 & 133851895 &  6694814 & 5\\
\hline
13 & 115169878 &  4976920 & 4.32\\
\hline
14 & 107349540 &  4516567 & 4.2\\
\hline
15 & 102531392 &  4031897 & 3.9\\
\hline
16 & 90354753 &  4058929 & 4.49\\
\hline
17 & 81195210 &  3761871 & 4.63\\
\hline
18 & 78077248 &  3884870 & 4.98\\
\hline
19 & 59128983 &  2636592 & 4.46\\
\hline
20 & 63025520 &  2981738 & 4.73\\
\hline
\end{tabular}
\captionsetup{justification=centering}
\caption{\label{tab:cont} Количество чтений, замапленных на каждую хромосому.}
\end{center}
\end{table} 

	\section{Геномные контексты}
	\label{contexts}

	Далее в таблицах ~\ref{cont3}, ~\ref{cont4},~\ref{cont5},~\ref{cont6} и ~\ref{cont8} предоставлены результаты работы первой части алгоритма по поиску наиболее склонных к ошибках контекстов. Были проведены запуски для контекстов длин 3, 4, 5, 6, 8 соответственно. В таблицах показаны наиболее значимые контексты с разностью долей ошибок $ERD \ge 0.001$, $ERD \ge 0.01$ и $ERD \ge 0.1$ в зависимости от длины контекста. Можно заметить, что контексты большей длины увеличивают разницу между вероятностями разносторонних ошибок. Это вызвано тем, что контекстная ошибка более специфична, и малой длины недостаточно для её явного выявления. Увеличивая длину, увеличивается пространство контекстов для поиска, и точнее охарактеризовывается последовательность.  Например, контекст $CGGGT$ является самым <<плохим>> контекстом длины пять, но из его возможного расширения до длины шесть (контексты $NCGGGT$, где $N$ -- любой нуклеотид) только контекст $GCGGGT$ присутствует в таблице контекстов длины шесть ~\ref{cont6} и также является самым <<плохим>> среди них. Получается, что остальные три контекста $ACGGGT$, $CCGGGT$, $TCGGGT$ не склонны к ошибкам, хотя они считаются <<плохими>> для длины пять.

\begin{table}[H]
\begin{center}
\resizebox{\textwidth}{!}{%
\begin{tabular}{|c|c|c|c|c|c|c|c|c|}
\hline
К-мер & Вхождения & Пр. сов. & Пр. несов. & Обр. сов. & Обр. несов. & Пр. ошибка & Обр. ошибка & Разность \\
\hline
$GGT$ & 60732854 & 41985781 & 42586008 & 565297 & 118547 & 0.013 & 0.003 & 0.01 \\
\hline
$AGT$ & 84204009 & 67999337 & 68419092 & 490541 & 210975 & 0.007 & 0.003 & 0.004 \\
\hline
$CGT$ & 13132547 & 8961784 & 8991629 & 51931 & 32347 & 0.006 & 0.004 & 0.002 \\
\hline
$TGT$ & 105296533 & 86717445 &  86882193 &  389189 & 229229 & 0.004 & 0.002 & 0.002 \\
\hline
\end{tabular}}
\end{center}
\captionsetup{justification=centering}
\caption{Статистика для контекстов длины три.}
\label{cont3}
\end{table} 

\begin{table}[H]
\begin{center}
\resizebox{\textwidth}{!}{%
\begin{tabular}{|c|c|c|c|c|c|c|c|c|}
\hline
К-мер & Вхождения & Пр. сов. & Пр. несов. & Обр. сов. & Обр. несов. & Пр. ошибка & Обр. ошибка & Разность \\
\hline
GGGT & 15196794 & 9460056 & 9665451 & 172636 & 25330 & 0.018 & 0.003 & 0.015 \\
\hline
AGGT & 20659881 & 14747122 & 14999554 & 215593 & 41287 & 0.014 & 0.003 & 0.011 \\ 
\hline
CGGT & 2626319 & 1512962 & 1516950 & 23800 & 7207  & 0.015 & 0.005 & 0.01 \\
\hline
GAGT & 18144417 & 13463587 & 13594955 & 141662 & 43941 & 0.01 & 0.003 & 0.007 \\
\hline
TGGT & 22249856 & 16265641 & 16404053 & 153268 & 44723 & 0.009 & 0.003 & 0.006 \\
\hline
AAGT & 25792322 & 22174554 & 22334986 & 168525 & 76986 & 0.008 & 0.003 & 0.005 \\
\hline
\end{tabular}}
\end{center}
\captionsetup{justification=centering}
\caption{Статистика для контекстов длины четыре.}
\label{cont4}
\end{table} 

\begin{table}[H]
\begin{center}
\resizebox{\textwidth}{!}{%
\begin{tabular}{|c|c|c|c|c|c|c|c|c|}
\hline
К-мер & Вхождения & Пр. сов. & Пр. несов. & Обр. сов. & Обр. несов. & Пр. ошибка & Обр. ошибка & Разность \\
\hline
CGGGT & 990114 & 442364 & 452276 & 12915 & 1721 & 0.028 & 0.004 & 0.024 \\
\hline
GGGGT & 3886664 & 2145220 & 2197336 & 48798 & 5921 & 0.022 & 0.002 & 0.02 \\
\hline
AGGGT & 4459507 & 2969088 & 3035824 & 59436 & 7893 & 0.02 & 0.003 & 0.017 \\
\hline
GAGGT & 5713420 & 3545840 & 3642448 & 65807 & 11105 & 0.018 & 0.003 & 0.015 \\
\hline
ACGGT & 797855 & 502430 & 501395 & 12170 & 4555 & 0.023 & 0.009 & 0.014 \\
\hline
GCGGT & 726432 & 332825 & 335080 & 5298 & 710 & 0.016 & 0.002 & 0.014 \\
\hline
CAGGT & 6000798 & 3864332 & 3937277 & 60452 & 10627 & 0.015 & 0.003 & 0.012 \\
\hline
AAGGT & 5126210 & 4185958 & 4233780 & 59217 & 12563 & 0.014 & 0.003 & 0.011 \\ 
\hline
TGGGT & 5860509 & 3903384 & 3980015 & 51487 & 9795 & 0.013 & 0.002 & 0.011 \\ 
\hline
GGAGT & 5459860 & 3349963 & 3383885 & 44776 & 9614 & 0.013 & 0.003 & 0.01 \\ 
\hline
CTGGT & 5317600 & 3611481 & 3648318 & 41274 & 8474 & 0.011 & 0.002 & 0.009 \\ 
\hline
AGAGT & 5990699 & 4834273 & 4892044 & 57677 & 17450 & 0.011 & 0.003 & 0.008 \\
\hline
\end{tabular}}
\end{center}
\captionsetup{justification=centering}
\caption{Статистика для контекстов длины пять.}
\label{cont5}
\end{table} 

\begin{table}[H]
\begin{center}
\resizebox{\textwidth}{!}{%
\begin{tabular}{|c|c|c|c|c|c|c|c|c|}
\hline
К-мер & Вхождения & Пр. сов. & Пр. несов. & Обр. сов. & Обр. несов. & Пр. ошибка & Обр. ошибка & Разность \\
\hline
GCGGGT & 251963 & 80429 & 87008 & 5272 & 291 &  0.062 & 0.003 & 0.059 \\
\hline
AACGGT & 150467 & 117374 & 116122 & 7766 & 3714 &  0.062 & 0.03 & 0.032 \\
\hline
GGCGGT & 166292 & 55775 & 58739 & 1919 & 148 &  0.033 & 0.003 & 0.03 \\
\hline
CGGGGT & 408673 & 146530 & 147834 & 4793 & 354 &  0.032 & 0.002 & 0.03 \\
\hline
GGGGGT & 840046 & 414738 & 431525 & 12632 & 1403 &   0.029 & 0.003 & 0.026 \\
\hline
GCAGGT & 1247617 & 741624 & 770163 & 21764 & 1959 &   0.027 & 0.002 & 0.025 \\
\hline
GGAGGT & 1720857 & 926138 & 957869 & 24750 & 3039 &   0.026 & 0.003 & 0.023 \\
\hline
GGCTGT & 1310515 & 792915 & 818194 & 18734 & 1989 &   0.023 & 0.002 & 0.021 \\
\hline
GAGGGT & 1030710 & 636174 & 658828 & 15192 & 1884 &   0.023 & 0.003 & 0.02 \\
\hline
CGAGGT & 394678 & 173761 & 179768 & 4264 & 640 &   0.024 & 0.004 & 0.02 \\
\hline
CACGTA & 262918 & 205672 & 205869 & 9025 & 4589 &   0.042 & 0.022 & 0.02 \\
\hline
\end{tabular}}
\end{center}
\captionsetup{justification=centering}
\caption{Статистика для контекстов длины шесть.}
\label{cont6}
\end{table} 

\begin{table}[H]
\begin{center}
\resizebox{\textwidth}{!}{%
\begin{tabular}{|c|c|c|c|c|c|c|c|c|}
\hline
К-мер & Вхождения & Пр. сов. & Пр. несов. & Обр. сов. & Обр. несов. & Пр. ошибка & Обр. ошибка & Разность \\
\hline
TCGAGTCA & 9022 & 6903 & 7226 & 2792 & 612 & 0.288 & 0.078 & 0.21 \\
\hline
GAACGGTT & 7945 & 6613 & 6641 & 1635 & 53 & 0.198 & 0.008 & 0.19 \\
\hline
TGGCGGGT & 34865 & 6191 & 7405 & 1278 & 29 & 0.171 & 0.004 & 0.167 \\
\hline
CGGCGGGT & 5779 & 608 & 792 & 112 & 2 & 0.156 & 0.003 & 0.153 \\
\hline
CGGCAGGT & 15983 & 4286 & 5164 & 751 & 25 & 0.149 & 0.005 & 0.144 \\
\hline
TGGCAGGT & 115629 & 46330 & 55014 & 7477 & 155 & 0.139 & 0.003 & 0.136 \\
\hline
CATTCGAA & 11890 & 15898 & 15990 & 4664 & 1610 & 0.227 & 0.091 & 0.135 \\
\hline
GACTCGAG & 9676 & 5754 & 5700 & 1744 & 616 & 0.233 & 0.098 & 0.135 \\
\hline
TGGCTGGT & 103487 & 44936 & 53871 & 7137 & 128 & 0.137 & 0.002 & 0.135 \\
\hline
CGGCTGGT & 16037 & 4611 & 5446 & 700 & 10 & 0.132 & 0.002 & 0.130 \\
\hline
CTCACCGA & 12545 & 12380 & 12177 & 1925 & 62 & 0.135 & 0.005 & 0.130 \\
\hline
CGCTTTGG & 22270 & 17049 & 16607 & 4614 & 1609 & 0.213 & 0.088 & 0.125 \\
\hline
GGCGGGGT & 29042 & 6195 & 7084 & 778 & 20 & 0.112 & 0.003 & 0.109 \\
\hline
AGGCGGGT & 84997 & 23347 & 27209 & 2612 & 94 & 0.1 & 0.003 & 0.097 \\
\hline
GTGGCTGT & 108175 & 48167 & 55280 & 5042 & 123 & 0.095 & 0.002 & 0.093 \\
\hline
\end{tabular}}
\end{center}
\captionsetup{justification=centering}
\caption{Статистика для контекстов длины восемь.}
\label{cont8}
\end{table} 

На рис. ~\ref{cont_error_example} показана позиция с явным присутствием склонности к ошибкам с одной стороны. Для  визуализации использовался \emph{Integrative Genomics Viewer, IGV} [], [].

	\begin{figure}[h!]
	\center{\includegraphics[width=1.05\textwidth]{cont_error_example}}
	\caption{Пример контекстных ошибок при $TGGCTGGT$. Красный цветом обозначены чтения, идущие в прямом направлении, красным --- в обратном. Показана часть чтений позиции 7936309 двадцатой хромосомы. Всего в прямом направлении 49 совпадений и 43 ошибки $G$, в обратном только 106 совпадений.}
	\label{cont_error_example}
	\end{figure}

Для наилучшей точности и в предположении о том, что малая часть контекстов индуцируют ошибки, были отобраны тринадцать наиболее ошибочных контекстов длины восемь с $ERD \ge 0.1$. Эти контексты показаны в таблице ~\ref{chosen_cont}.

\begin{table}[H]
\begin{center}
\begin{tabular}{|c|c|}
\hline
Контексты & ERD \\
\hline
TCGAGTCA & 0.21 \\
\hline
GAACGGTT &  0.19 \\
\hline
TGGCGGGT &  0.167 \\
\hline
CGGCGGGT & 0.153 \\
\hline
CGGCAGGT & 0.144 \\
\hline
TGGCAGGT & 0.136 \\
\hline
CATTCGAA &  0.135 \\
\hline
GACTCGAG &  0.135 \\
\hline
TGGCTGGT &  0.135 \\
\hline
CGGCTGGT & 0.130 \\
\hline
CTCACCGA & 0.130 \\
\hline
CGCTTTGG & 0.125 \\
\hline
GGCGGGGT & 0.109 \\
\hline
\end{tabular}
\end{center}
\captionsetup{justification=centering}
\caption{Выбранные контексты с разностью вероятности ошибки с разных сторон.}
\label{chosen_cont}
\end{table} 

Суммарное число вхождений всех этих контекстов не превышает и десятой доли процента для каждой хромосомы. Но в абсолютном значении это количество потенциальных мест на присутствие большого скопления ошибок секвенирования, и оно достаточно большое, чтобы повысить эффективность алгоритмов поиска мутаций в конечном итоге. Подробее показано в таблице ~\ref{cont_occur}.

\begin{table}[H]
\begin{center}
\begin{tabular}{|c|c|c|c|}
\hline
Номер хромосомы & Длина & Вхождения & \% \\
\hline
1 & 249250621 &  99988 & 0.04\\
\hline
2 & 243199373 &  98676  & 0.04\\
\hline
3 & 198022430 &  77766  & 0.039\\
\hline
4 & 191154276 &  69658  & 0.036\\
\hline
5 & 180915260 &  69917  & 0.038\\
\hline
6 & 171115067 &  66212  & 0.038\\
\hline
7 & 159138663 &  66578  & 0.041\\
\hline
8 & 146364022 &  58370  & 0.039\\
\hline
9 & 141213431 &  52373  & 0.037\\
\hline
10 & 135534747 &  57691  & 0.042\\
\hline
11 & 135006516 &  56392  & 0.041\\
\hline
12 & 133851895 &  55077  & 0.041\\
\hline
13 & 115169878 &  35968 & 0.031\\
\hline
14 & 107349540 &  37631  & 0.035\\
\hline
15 & 102531392 &  37064  & 0.036\\
\hline
16 & 90354753 &  41000  & 0.045\\
\hline
17 & 81195210 &  41942  & 0.051\\
\hline
18 & 78077248 &  30040  & 0.038\\
\hline
19 & 59128983 &  33961 & 0.057\\
\hline
20 & 63025520 &  29563  & 0.047\\
\hline
\end{tabular}
\end{center}
\captionsetup{justification=centering}
\caption{Количество вхождений выбранных контекстов в каждую хромосому.}
\label{cont_occur}
\end{table} 


\section{Выбор геномных позиций}
\label{genomepos}
	
	Далее были отобраны удовлетворяющие нашим условия геномные позиции, на основе которых будет собираться статистика для контекстов чтений. В среднем было отобрано по 17 \% у каждой хромосомы.

\begin{table}[H]
\large
\begin{center}
\begin{tabular}{|c|c|c|c|}
\hline
Номер хромосомы & Длина & Позиции & \% \\
\hline
1 & 249250621 &  42379457 & 17\\
\hline
2 & 243199373 &  47137187  & 19.38\\
\hline
3 & 198022430 &  39654565  & 20.03\\
\hline
4 & 191154276 &  38920699  & 20.36\\
\hline
5 & 180915260 &  35810456  & 19.79\\
\hline
6 & 171115067 &  33371937  & 19.5\\
\hline
7 & 159138663 &  29581306  & 18.59\\
\hline
8 & 146364022 &  28299277  & 19.33\\
\hline
9 & 141213431 &  21491835  & 15.22\\
\hline
10 & 135534747 &  24405500  & 18\\
\hline
11 & 135006516 &  24955988  & 18.49\\
\hline
12 & 133851895 &  25381024  & 18.96\\
\hline
13 & 115169878 &  19601846 & 17.02\\
\hline
14 & 107349540 &  17217889  & 16.04\\
\hline
15 & 102531392 &  14970590  & 14.6\\
\hline
16 & 90354753 &  13176982  & 14.58\\
\hline
17 & 81195210 &  12211382  & 15.04\\
\hline
18 & 78077248 &  14936698  & 19.13\\
\hline
19 & 59128983 &  7432700 & 12.57\\
\hline
20 & 63025520 &  10427796  & 16.55\\
\hline
\end{tabular}
\label{cont} 
\end{center}
\captionsetup{justification=centering}
\caption{Количество отобранных позиций для каждой хромосомы.}
\end{table} 

\section{Контексты чтений}

Учитывая отобранные позиции в геноме, была собрана статистика для каждого контекста длины восемь и качества его последнего нуклеотида. На рис. ~\ref{phred_graph} заметно, что при качестве от 15 до 40 секвенатор сильно занижает оригинальные качества, а от 2 до 15 немного завышает. Особое внимание надо уделить качеству 2, которое, возможно, секвенатор присвает к очень зашумленным данным. Оно совсем не соответствует действительности, на практике ситуация гораздо лучше, чем оценивает её секвенатор. 

	\begin{figure}[H]
	\center{\includegraphics[width=1.05\textwidth]{phreds}}
	\caption{График соответствия показателей качества нуклеотидов, присвоенных секвенатором, и посчитанных в зависимости от контекста. Точки --- контексты. Интенсивность отражает количество вхождений контекста во все чтения.}
	\label{phred_graph}
	\end{figure}

Рассмотрим геномный контекст $TGGCTGGGT$, в последнем элементе которого склонны скапливаться ошибки. Найдем контекст в чтениях, который вызывает эти ошибки. Возможно четыре соответствующих ему контекста: $TGGCTGGGT$, $TGGCTGGGG$, $TGGCTGGGA$, $TGGCTGGGC$. Посмотрев их статистику, легко сказать, что только у $TGGCTGGGG$ заявленные качества совсем не соответствуют посчитанным (рис. ~\ref{bad_plot}).

	\begin{figure}[h!]
	\center{\includegraphics[width=1.05\textwidth]{bad_plot}}
	\caption{График распределения качества нуклеотидов после контекста чтения $TGGCTGGG$. Все точки находятся намного ниже диагонали.}
	\label{bad_plot}
	\end{figure}

На рис. ~\ref{diag} приведена диаграмма распределения геномных нуклеотидов в позициях, которые соотвествуют последнему элементу контекста  $TGGCTGGG$ в чтениях. Заметно, что для всех промежутков высока вероятность ошибки замены T на G, то есть секвенатор независимо от присвоенного качества после последовательности нуклеотидов $TGGCTGG$ неверно прочитывает настоящий нуклеотид $T$ в сторону $G$, тем самым вызывая скопления ошибок, и как следствие, ложный гетерозиготный генотип в соответствующей позиции. В таблице ~\ref{bad_table} показано подробное распределение нуклеотидов после этого контекста $TGGCTGGG$ для каждого показателя качества последнего элемента $G$.

	\begin{figure}[h!]
	\center{\includegraphics[width=1.05\textwidth]{diag}}
	\caption{Диаграмма распределения референсных нуклеотидов, соответствующих последнему нуклеотиду $G$ контекста $TGGCTGGG$.}
	\label{diag}
	\end{figure}

\begin{table}[!h]
\large
\begin{center}
\resizebox{\textwidth}{!}{
\begin{tabular}{|M{2.5cm}|M{2.5cm}|M{1.5cm}|M{1.5cm}|M{1.5cm}|M{1.5cm}|M{2.5cm}|M{2.5cm}|}
\hline
Phred-качество секвенатора & Вероятность ошибки по Phred & A & C & T & G & Вероятность ошибки & Вероятность T\\
\hline
2.0 & 0.631 & 138 & 49 & 330 & 1301 & 0.2846 & 0.1815 \\ \hline
7.0 & 0.1995 & 6 & 1 & 32 & 1 & 0.9524 & 0.8 \\ \hline
8.0 & 0.1585 & 13 & 3 & 26 & 3 & 0.9149 & 0.5778\\ \hline
9.0 & 0.1259 & 8 & 2 & 30 & 5 & 0.8723 & 0.6667\\ \hline
10.0 & 0.1 & 47 & 15 & 147 & 12 & 0.9417 & 0.6652\\ \hline
11.0 & 0.0794 & 73 & 16 & 224 & 15 & 0.9515 & 0.6829\\ \hline
12.0 & 0.0631 & 57 & 18 & 157 & 39 & 0.8535 & 0.5793\\ \hline
13.0 & 0.0501 & 34 & 3 & 100 & 23 & 0.8519 & 0.625\\ \hline
14.0 & 0.0398 & 10 & 5 & 91 & 34 & 0.7535 & 0.65\\ \hline
15.0 & 0.0316 & 5 & 1 & 114 & 21 & 0.8462 & 0.8085\\ \hline
16.0 & 0.0251 & 5 & 1 & 110 & 51 & 0.6923 & 0.6587\\ \hline
17.0 & 0.02 & 2 & 0 & 135 & 112 & 0.5498 & 0.5422\\ \hline
18.0 & 0.0158 & 2 & 0 & 188 & 91 & 0.6749 & 0.669\\ \hline
19.0 & 0.0126 & 3 & 0 & 123 & 1849 & 0.0642 & 0.0623\\ \hline
20.0 & 0.01 & 0 & 1 & 195 & 46 & 0.8074 & 0.8058\\ \hline
21.0 & 0.0079 & 0 & 0 & 126 & 93 & 0.5747 & 0.5753\\ \hline
22.0 & 0.0063 & 0 & 0 & 117 & 25 & 0.8194 & 0.8239\\ \hline
23.0 & 0.005 & 0 & 0 & 162 & 82 & 0.6626 & 0.6639\\ \hline
24.0 & 0.004 & 0 & 0 & 88 & 62 & 0.5855 & 0.5867\\ \hline
25.0 & 0.0032 & 0 & 0 & 83 & 106 & 0.4398 & 0.4392\\ \hline
26.0 & 0.0025 & 0 & 0 & 113 & 158 & 0.4176 & 0.417\\ \hline
27.0 & 0.002 & 0 & 0 & 68 & 242 & 0.2212 & 0.2194\\ \hline
28.0 & 0.0016 & 0 & 0 & 66 & 368 & 0.1537 & 0.1521\\ \hline
29.0 & 0.0013 & 0 & 0 & 70 & 302 & 0.1898 & 0.1882\\ \hline
30.0 & 0.001 & 0 & 0 & 81 & 531 & 0.1336 & 0.1324\\ \hline
31.0 & 0.0008 & 0 & 0 & 136 & 797 & 0.1465 & 0.1458\\ \hline
32.0 & 0.0006 & 0 & 0 & 84 & 947 & 0.0823 & 0.0815\\ \hline
33.0 & 0.0005 & 0 & 0 & 113 & 652 & 0.1486 & 0.1477\\ \hline
34.0 & 0.0004 & 0 & 0 & 61 & 640 & 0.0882 & 0.087\\ \hline
35.0 & 0.0003 & 0 & 0 & 68 & 1724 & 0.0385 & 0.0379\\ \hline
36.0 & 0.0003 & 0 & 0 & 94 & 1415 & 0.0629 & 0.0623\\ \hline
37.0 & 0.0002 & 0 & 0 & 44 & 3392 & 0.0131 & 0.0128\\ \hline
38.0 & 0.0002 & 0 & 0 & 17 & 3112 & 0.0057 & 0.0054\\ \hline
39.0 & 0.0001 & 0 & 0 & 25 & 41742 & 0.0006 & 0.0006\\ \hline
40.0 & 0.0001 & 0 & 0 & 1 & 1183 & 0.0008 & 0.0008\\ \hline
\end{tabular}}
\end{center}
\captionsetup{justification=centering}
\caption{Распределение геномных нуклеотидов при контексте чтений $TGGCTGGG$.}
\label{bad_table}
\end{table} 

\section{Замена качества в чтениях и анализ эффекта}

	Замена (перекалибровка) показателей качества, присвоенных секвенатором, должна повышать эффективность поиска ОНП. Показателем эффективности чаще всего является отношение ложноположительных мутаций ко всем найденным, так называемая частота ложных обнаружений (False Discovery Rate, FDR). Для валидации был выбран размеченный набор мутаций двадцатой хромосомы, который был специально подготовлен консорциумом Genome in a Bottle [ref], организованный Национальным институтом стандартов и технологий (National Institute of Standards and Technology, NIST [ref]), версии 2.18 [ref]. Этот набор был создан с помощью объединения 14 различных библиотек чтений, полученных на пяти различных технологиях секвенирования, и удалением участков с неразрешимыми различиями или с наличием признаков дизбаланса, описанного в предыдущей главе.
	Для сравнения были выбраны следующие данные:
\begin{itemize}
\item оригинальный набор чтений 20-ой хромосомы;
\item набор чтений с заменой только тех качеств, которые лучше оригинальных;
\item набор чтений с заменой всех качеств нуклеотидов на рассчитанные;
\item набор чтений с просто заменой всех качеств на максимальные, которые может присвоить секвенатор.
\end{itemize}	
	Для поиска возможных мутаций был выбран метод \emph{Unified Genotyper}, входящий в набор инструментов по анализу генома (\emph{The Genome Analysis Toolkit}, \emph{GATK} []), разработанных институтом Броуда (\emph{Broad Institute} []).
	Результаты эксперимента приведены в таблице ~\ref{recalib_results}. Точность показывает долю по-настоящему верных мутаций относительно всего количества кандидатов и вычисляется как $\frac {TP} {TP + FP}$, полнота --- долю действительно верных среди всех известных верных в тестовой выборке и равна $\frac {TP} {TP + FN}$. F-мера [] является гармоническим средним между полнотой и точностью и равна $ 2 \cdot \frac {{Precision} \times {Recall}} {{Precision} + {Recall}}$. Значения меры находятся в промежутке между нулем и единицей; чем больше значение, тем лучше.

По результатам самое большое количество истинно-положительных мутаций достигается при замене всех качеств нуклеотидов на максимально возможное для секвенатора, то есть на $phred$ равным сорока для секвенаторов $Illumina$. По сути все нуклеотиды в чтениях становятся равнозначными и решающую роль начинает играть их количество (соотношение). В таком случае небольшое наличие одинаковых нуклеотидов, которые отличаются от референсного, заставляет воспринимать их как мутацию. Естественно, будет заявлено больше подозрительных позиций, некоторая часть которых действительно будет являться настоящими мутациями, что и показал эксперимент. Но вместе с ними будет и больше ложно-положительных (мнимых) мутаций, что крайне нежелательно. В итоге такая замена имеет наибольшую полноту и наименьшую точность. В то же время $F_1$-мера тоже максимальна, но в нашем случае точность и полнота не равнозначны. Важнее точность или аналогичная ей величина $ FDR = 1 - $ точность.

Замена только на улучшающие показатели качества, то есть на такие показатели, которые больше оригинальных, увеличивает как точность, так и полноту. Такая замена добавляет некоторое количество настоящих мутаций, не увеличивая, а в данном случае даже уменьшая количество мнимых. Это показывает, что метод правильно расчитывает улучшающие показатели качества (как минимум) и позволяет немного, но повысить содержание настоящих мутаций.   

Замена всех показателей (как улучшающих, так и ухудшающих) дает еще большую полноту, сохраняя точность, чем просто замена на улучшающие. Такой способ имеет чуть меньшую $F_1$-меру, чем при замене на константу, но гораздо большую точность. Это показывает, что замена показателей качества согласно предложенному методу эффективнее, чем уравнивание (исключение) их. А так же доказывает, что показатели качества важны для эффективного поиска мутаций, оригинальные показатели секвенатора не соответствуют действительности и их можно улучшить, тем самым повысив эффективность поиска.

\begin{table}[!h]
\begin{center}
\huge
\resizebox{\textwidth}{!}{
\begin{tabular}{|M{3.0cm}|M{2.5cm}|M{2.2cm}|M{1.5cm}|M{1.5cm}|M{1.5cm}|M{2.7cm}|M{2.7cm}|M{2.0cm}|M{1.3cm}|}
\hline
Данные & Всего & TP & FP & FN & TN & Точность, \% & Полнота, \% & F1-мера  & FDR \\ \hline
Ориг. & 102240 & 61633 & 614 & 13649 & 26344 & 99.01 & 81.87 & 0.8963 & 0.99\\ \hline
% Замена всех & 102240 & 61869 & \textbf{612} & 13413 & 26346 & 99.02 & 82.18 & 0.8982 & 0.98\\ \hline
% Замена на лучшее & 102240 & 63610 & 624 & 11672 & 26334 & \textbf{99.03} & 84.5 & 0.9119 & \textbf{0.97}\\ \hline
Замена на лучшее & 102240 & 62568 & \textbf{612} & 12714 & 26346 & 99.03 & 83.11 & 0.9038 & 0.97\\ \hline
Замена всех & 102240 & 63641 & 621 & 11641 & 26337 &  \textbf{99.03} & 84.54 & 0.9121 &  \textbf{0.97}\\ \hline
Замена на константу & 102240 & \textbf{63902} & 697 & 11380 & 26261 & 98.92 & \textbf{84.88} & \textbf{0.9137} & 1.08\\ \hline
\end{tabular}}
\end{center}
\captionsetup{justification=centering}
\caption{Результаты эксперимента.}
\label{recalib_results}
\end{table} 

\section{Выводы по главе \protect\ref{chapter3}}

Приведены результаты основных частей метода. Приведен пример, что контекстно-зависимые ошибки секвенирования действительно имеют место быть в реальных данных секвенатора. Предложенный метод успешно находит такие контексты и учитывает их при подсчете статистики для расчета вероятностей ошибок. Было показано, что показатели качества нуклеотидов важны для эффективного поиска мутаций, и замена их согласно построенной статистической модели позволяет повысить эту эффективность. 

