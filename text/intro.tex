\startprefacepage


Процесс определения нуклеотидной или аминокислотной последовательности биополимеров (белков и нуклеиновых кислот) известен как секвенирование. В результате получают в текстовом виде формальное описание их первичной структуры в виде последовательности мономеров. Многие современные задачи биологии и медицины требуют знания генома живых организмов. Генной инженерии, исследованию наследственных заболеваний и многим другим задачам нужен быстрый, дешевый и качественный способ безошибочного определения измененных участков ДНК.

Технологии секвенирования нового поколения позволяют прочитать единовременно несколько участков генома, что является главным отличием от предыдущих технологий. Секвенаторы нового поколения стали значительно дешевле и гораздо эффективнее своих предшественников. На сегодняшний день производительность некоторых секвенаторов измеряется уже сотнями миллиардов пар оснований, что, например, позволяет подобным приборам сканировать индивидуальный геном человека всего за несколько дней.

Наличие ошибок в определении правильного нуклеотида в чтении ДНК является общей проблемой для всех технологий секвенирования. К сожалению, секвенаторы на основе последних технологий, как показывает практика, более подвержены ошибкам, чем предыдущие. Последствия таких ошибок зависят от конкретного применения,
начиная от незначительных информационных неприятностей, заканчивая крупными проблемами, влияющими
на биологические выводы. Поэтому требуются надежные показатели качества данных.

Современные секвенаторы вместе с прочитанным нуклеотидом предоставляют показатель качества, отображающий вероятность его ошибочного определения, говоря другими словами, уверенность секвенатора в своем выборе. Такой показатель основан на физических характеристиках процесса. 
% В зависимости от фундаментального метода, заложенного в ту или иную технологию, это может быть \emph{todo: % перечислить}. 
К сожалению, эти показатели не всегда описывают настоящее состояние дел. Такие показатели особенно важны для поиска мутаций. Тем не менее, в конкретных случаях, может быть получено большое количество неверно распознанных мутаций. Нужно уметь различать позиции с настоящими  мутациями и позиции с ошибками секвенирования.

В настоящее время различают несколько видов ошибок секвенирования. Одним из них является вид систематических ошибок, возникающих вследствие наличия определенной последовательности нуклеотидов перед ними.  Такие ошибки, заисимые от контекста, имеют свойство скапливаться в некоторой геномной позиции ввиду того что, большинство чтений,  соответствующие данной позиции, будут содержать одинаковые <<плохие>> фрагменты. Такое ошибочное скопление пагубно влияет на алгоритмы поиска мутаций. По факту имеем большое содержание в реальности несуществующего нуклеотида с высоким качеством, предоставленным секвенатором, в определенной геномной позиции, которое логично объяснить только мутацией.

Идея метода, разработанного в данной работе, заключается в построении неявной статистической модели распределения систематических ошибок секвенирования в засимости от контекста. Она позволяет оценить показатель качества нуклеотидов, возникщих благодаря ошибкам, показать насколько нельзя им доверять в конкретной геномной позиции. Метод повышает эффективность алгоритмов поиска мутаций, которые в большинстве своем основаны на анализе показателей качества, благодаря замене заявленных качеств на вычисленные, отражающие всю суть возникшей ситуации с ними.
