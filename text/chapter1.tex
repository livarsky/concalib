\chapter{Обзор предметной области}
\label{chapter1}

В данной главе будет дан обзор основных элементов предметной области. Будет сформулирована задача определения вероятности ошибки на основе её контекста, будут описаны проблемы распознавания ошибок секвенирования от мутаций и существующие способы для их различия.

\section{Ген и геном}
\label{genomic}

Ген — структурная и функциональная единица наследственности, участок молекулы ДНК, где закодирована информация о синтезе одной полипептидной цепи с определенной аминокислотной последовательностью. Контролируют все клеточные процессы на молекулярном уровне, обеспечивая синтез белков. Если белок состоит из более чем одной полипептидной цепи, синтез каждой из них контролируется самостоятельным геном. Ген и признак организма не связаны простым соотношением. Все сложные признаки (например, способность слышать) контролируются многими генами. Вместе с тем один ген способен оказывать влияние на развитие сразу нескольких признаков. Ген может существовать в нескольких формах (аллелях), определяющих различные варианты контролируемого им признака (например, цвет глаз).

Геном — совокупность наследственного материала (генов), сосредоточенного в гаплоидном наборе хромосом некоторого живого организма. Развитие и функционирование всех известных живых существ определяется их геномом. Кроме того, геном определяет то, каким может получиться потомство того или иного существа. Различия между разными видами, как и между разными особями одного вида, возникают в результате того, что у них различные гены. 

\section {Строение ДНК}

Нуклеотид — сложное вещество,  состоящее из азотистого основания, сахара (дезоксирибозы) и фосфатной группы. Являются основой для построения ДНК: цепи ДНК состоят из последовательности нуклеотидов. В ДНК бывает четыре вида оснований: аденин, гуанин, тимин, цитозин.

ДНК (Дезоксирибонуклеиновая кислота) — макромолекула, обеспечивающая хранение генетической информации. Структура молекулы представляется в виде двойной спирали. Нуклеотиды в каждой из двух цепей соединяются фосфодиэфирными связями. Сами эти цепи держатся вместе за счет водородных связей, возникающих между азотистыми основаниями, которые находятся друг напротив друга в цепях. Водородные связи возникают согласно принципу комплементарности: аденин связывается только с тимином, гуанин — с цитозином. Из этого следует что, зная последовательность нуклеотидов в одной цепи,  можно абсолютно точно установить последовательность в другой. 
Эти последовательности обратно комплементарны друг другу. В каждой из них есть свое направление, и эти направления противоположны друг другу.

\section {Генетический код}

Генетический код — способ записи структуры белков в молекулах нуклеиновых кислот в виде последовательности нуклеотидов. Каждый белок состоит из последовательности аминокислот. Существует 20 основных аминокислот, которые могут входить в состав белков. Из четырех возможных нуклеотидов можно составить 64 различных "слов" длины три. Такие трёхбуквенные слова называется кодоны или триплеты. 61 кодон из 64 возможных кодирует определенные аминокислоты. Некоторые аминокислоты кодируются несколькими кодонами, так как число кодирующих кодонов больше количества основных аминокислот. Это свойство называется вырожденностью кода. Другое свойство, при котором один кодон входит в состав только одной аминокислоты, называется специфичностью или однозначностью кода. Вдобавок, код не перекрывается, считывается последовательно в одной направлении. И, наконец, он единый для всех живых существ. Это самое удивительное свойство — универсальность. Расшифровка кода была осуществлена в 1961-1965 гг. и является одним из ярких достижений биологии.

\section {Репликация}
Репликация — процесс синтеза дочерней ДНК на матрице родительской ДНК. Во время клеточного деления связи между цепями ДНК разрушаются, и нити разделяются. Потом на каждой из них строится комплементарная дочерняя цепь. В результате в обеих клетках получается по одной полной копии молекулы ДНК. Идентичность требуется, чтобы полностью передать всю информацию, которая заключена в геноме клетки. На самом деле ДНК удваивается не последовательно, а отдельными участками  — репликонами. Они синтезируется независимо. Благодаря этому, репликация ДНК выполняется параллельно в нескольких местах, обеспечивая высокую скорость копирования.  

\section {Передача генома по наследству}
Сложные организмы размножаются половым путем. Обычно клетки таких организмов содержат диплойдный набор хромосом: по одному набору хромосом от каждого из родителей. При половом размножении такие организмы производят гаметы --- половые клетки, обладающие гаплоидным набором хромосом. Такие клетки создаются в результате мейоза. Мейоз --- вид клеточного деления, при котором из диплойдных клеток образуются гаплоидные гаметы. В профазе мейоза происходит кроссинговер --- перекомбинация, обмен частями гомологичных хромосом.  Поэтому гаплойдный набор половой клетки является комбинацией геномов, которые получены от каждого из родителей особи. В зиготе происходит объединение таких клеток, благодаря чему число хромосом вновь становится диплоидным. В итоге каждая такая клетка, полученная в результате слияния двух половых клеток родителей, содержит по одному "смешанному" геному от каждого из родителей. Таким образом поддерживается генетическое разнообразие.

\section {Мутации}
Мутация — наследуемое преобразование генотипа, которое может вызвать изменение свойств и признаков живого организма. Мутации могут случаться естественно, а могут возникать под воздействием внешней среды, индуцироваться различными факторами — мутагенами. 

Можно разделить мутации по следующим признакам:
\begin{itemize}
\item по масштабу воздействия (ген, хромосома, геном);
\item по месту происхождения (соматические или половые клетки);
\item по проявлению (доминантные или рецессивные);
\item по влиянию (полезные или вредные (летальные);
\item по причине возникновения (индуцируемые или спонтанные). 
\end{itemize}

Точечные (генные) мутация связана с изменением участка ДНК, чаще всего одного гена. Причинами могут быть удаление, вставка или замена пары нуклеотидов. Замены классифицируют на:
\begin{itemize}
\item транзиции --- пурин замещается на пурин (например. $A$ на $G$), либо пиримидин на пиримидин ($T$ на $С$);
\item трансверсии --- пурин замещается на пиримидин, либо наоборот.
\end{itemize}

Транзиции происходят в два раза чаще, чем трансверсии, благодаря схожести по химическим свойствам заменяемых нуклеотидов. Такой вид замен не сильно вляет на структуру ДНК и на функционирование и работоспособность всего организма в конечном итоге.

Хромосомная мутация (аберрация) заключается в нарушении структуры хромосом в связи с удвоением (дупликация), потерей (делеция), перемещением (транслокацией) их отдельных частей. Могут возникать как в одной хромосоме, так и между гомологичными и негомологичными.

Геномная мутация состоит в изменении числа хромосом. В результате неверного клеточного деления в хромосомном наборе может присутствовать лишняя хромосома, или, наоборот, отсутствовать.

В большинстве случаев мутации, ухудшающие работу клетки, приводят к её уничтожению. В стабильной среде обитания организмы содержат почти оптимальный генотип, поэтому большинство мутаций вызывают нарушение функций организма и приводят к смерти. Поэтому такие вредные мутации не распространяются дальше по поколениям. В редких случаях мутация может оказаться положительной и стать причиной появления полезных признаков. Такие особи получают преимущество и оказываются более приспособленными к условиям внешней среды и, как следствие, более живучими. Таким образом, мутации являются частью естественного отбора.

Для защиты от влияния вредной внешней среды и сопутствующих мутаций организм и клетки, в частности, вырабатывают механизмы противоборства им. Благодаря полученному иммунитету, такие клетки могут выживать под сильным воздействием мутагенов (например,  воздействие ультрафиолетового излучения или высокой температуры).

\section{Секвенаторы последнего поколения}

Благодаря разнообразным улучшениям, многочисленным модификациям и увеличению производительности вычислительной техники были разработаны технологии секвенирования, которые основываются на единовременном <<прочтении>> (распараллеливании) нескольких участков генома. В связи с этим секвенаторы на основе таких технологий стали намного эффективнее по скорости и стоимости, чем предыдущие. Производительность таких приборов настолько высока, что, например, лишь за несколько дней позволяет получить индивидуальный геном человека.

На рынке существует достаточно большое количество секвенаторов, основанных на различных принципах определения нуклеотидов. Поэтому все они различаются в скорости и стоимости секвенирования, стоимости самого прибора, сопутствующих типах ошибок и их количестве. Характеристики некоторых из них приведены в таблице ~\ref{seqtable}. 

\begin{table}[H]
\large
\begin{center}
\resizebox{\textwidth}{!}{
\begin{tabular}{|M{2.5cm}|M{2cm}|M{2cm}|M{2cm}|M{2cm}|M{2cm}|M{2cm}|}
\hline
Платформа & Illumina  MiSeq & Illumina GAIIx & Illumina HiSeq 2000 & Ion Torrent PGM &  PacBio RS &  454 GS FLX \\
\hline
Принцип & SBS (sequncing-by-synthesis) &  SBS &  SBS & Ионный полупроводник & Real-time & Пиросекве-нирование \\
\hline
Стоимость прибора & \$ 128k & \$ 256k & \$ 654k & \$ 80k & \$ 695k & \$ 500k \\
\hline
Время работы за цикл & 27 ч. & 10 д. &  11 д. & 2 ч. & 2 ч. &  23 ч.\\
\hline
Длина чтений & до 150 нукл. & до 150 нукл. & до 150 нукл. & $\sim$200 нукл. & $\sim$2500 нукл. & 700 нукл. \\
\hline
Количество нуклиотидов за цикл &  1.5-2 Гн & 30 Гн & 600 Гн & 20-1000 Мн  &  100 Мн & 700 Мн \\
\hline
Стоимость 1Г нукл. &  \$ 502 & \$ 148 & \$ 41 & \$ 1000 &  \$ 2000 & \$ 7000 \\
\hline
Доля ошибок & 0.8 \% & 0.76\%  & 0.26\% & 1.71 \% & 12.86 \% & 0.49 \% \\
\hline
Преимущ. тип ошибок & Замена & Замена & Замена & Вставка/ \newline удаление & Вставка/ \newline удаление & Вставка/ \newline удаление \\
\hline
\end{tabular}}
\end{center}
\captionsetup{justification=centering}
\caption{Сравнительная характеристика известных секвенаторов.}
\label{seqtable}
\end{table} 

\section{Ошибки секвенирования}
	Во всех платформах последнего поколения для считывания нуклеотидных последовательностей за более высокую скорость и низкую стоимость обработки данных приходится расплачиваться более частыми ошибками секвенирования. Но кроме обычных случайных несовпадений существуют еще и систематические источники ошибок. Поскольку любая ошибка в определении нуклеотида может быть расценена как мутация, исправление как можно большего количества ошибок любого типа является жизненно важным. Нужны хорошие статистические методы, которые бы различали высокую вероятность ошибки и гетерозиготный генотип в позициях с низким покрытием в чтениях. 
	В некоторых геномных позициях  во многих чтениях случаются скопления несовпадений с референсным нуклеотидом. Известно, что ошибки происходят ближе к концам чтений [?] и от влияния окружающих её мотивов последовательности. Например, ошибки склонны к появлению перед <<GG>> или после некоторого количества <<GGC>> [?], но, несмотря на предшествующие мотивы, ошибки наиболее склонны к концу чтений. Было обнаружено [? this], что есть геномные позиции, в которых ошибки случаются намного чаще, чем это может быть объяснено описанными эффектами. Такие ошибки имеют систематический характер появления. Было замечено, что в местах скопления систематических ошибок, как правило, ошибки появляются только с одной стороны секвенирования (прямая или обратная). Эта тенденция была замечена в [?], где направление возникновения ошибок использовалось для разделения мнимых мутаций и гетерозиготных позиций (настоящих мутаций). Возможным объяснением этому может служить, что в процессе секвенирования некоторых мотивов ДНК (различных в противоположных направлениях) возрастает вероятность ошибки в правильном определении нуклеотида. Это согласуется с известным наложением спектра поглощения <<G>> и <<T>> каналов, определяемых одним лазером в $Illumina$.
	В дальнейшем, систематические, зависимые от последовательности ошибки будут называться контекстными ошибками, а мотивы последовательности, которые индуцируют ошибки --- контекстами. Для отличия контекстных ошибок от настоящих мутаций будет использоваться дизбаланс количества ошибок в разных направлениях: поскольку ошибки  вызываются предшествующими мотивами, а не последующими, ошибка должна присутствовать с одной стороны и отсутствовать с другой.
	Благодаря предыдущим работам [?,?], обнаружение и отбор позиций с таким дизбалансом стало обычным этапом в обработке данных. The Genome Analysis Toolkit (GATK) [?, ?], для примера, вычисляет ко всем предполагаемым мутациям $p$-значение, полученное из теста для определения независимости несовпадений от направления чтений (точный тест Фишера). Такой метод, однако, требует достаточного покрытия чтениями [this]: если покрытие слабое, статистическая сила для такого определения слаба.
	В работе [this] было предложено объединять позиции и ошибки в них на основе их геномного контекста. При проверке на существенный дизбаланс для вычисления $p$-значения используется собранная статистика по всем позициям с таким контекстом, а не только ошибки в текущей единственной позиции. Такой <<общий взгляд>> на все позиции компенсирует общее низкое покрытие и держит статистическую силу на высоком уровне. К сожалению, такой метод позволяет только выделять наиболее склонные к ошибкам контексты относительно других, но не определять вероятность ошибки после них. 
На практике было бы полезнее иметь некую количественную характеристику (на сколько после каждого контекста всё плохо).

\section{Форматы хранения генетической информации}

Данные секвенатора нужно сохранять в удобном, простом, логичном формате. Вместе с прочитанными нуклеотидами секвенатор может выдавать дополнительные характеристики к ним. Например,  показатели качества прочтения нуклеотидов. Всё это надо правильно сохранять для удобства обработки.

\subsection{Формат FASTA}
FASTA — наиболее распространенный формат представления последовательностей нуклеотидов.
Формат FASTA представляет собой обычным текстовый файл. Опишем кратко формат:
\begin{itemize}
\item Первая строка начинается с символа <<$>$>> или <<$;$>>. Она содержит идентификаторы библиотеки чтений и самого чтения;
\item Следующие строки, которые начинаются опять же с <<$;$>> или <<$>$>>, содержат комментарии. 
\item Далее, в оставшихся строках идет сама последовательность нуклеотидов в виде латинских букв из алфавита \{$A$, $C$, $G$, $T$, $U$, $N$\}, которые соответствует аденину, цитозину, гуанину, тимину и урацилу.  Символ $N$ обозначает неизвестный нуклеотид.
\end{itemize}

\subsection{Формат FASTQ}
Формат FASTA не предусматривал хранение дополнительной информации к каждому чтению. В FASTQ это поправили.
Этот формат в выходных данных секвенатора, так как вместе с самой последовательностью предоставляет вероятность ошибки в выборе каждого элемента последовательности. Опишем формат:
\begin{itemize}
\item Для хранения каждого чтения отведено по четыре строки;
\item Первая строка начинается с символа <<@>>. Описывает то же, что и в FASTA;
\item Вторая строка описывает нуклеотидную последовательность в том же виде как и в FASTA;
\item Третья строка служит разделителем и содержит <<$+$>>;
\item Четвертая строка содержит для каждого нуклеотида второй строки некоторый символ, соответствующий его показателю качества. 
\end{itemize}

Для расчета показателя качества используется следующая формула:
	\begin{equation*}
	Q = -10 \cdot log_{10} p,
	\end{equation*}
где p — вероятность ошибки в определении нуклеотида в некоторой позиции. 

Теперь чтобы получить $ASCII$-код символа, нужно округлить $Q$ до ближайщего целого и прибавить константу.
Константа чаще всего равна 33 или 64 в зависимости от секвенатора.

\section{Формулировка задачи}
\label{Problem formulation}

По данному набору чтений в формате FASTQ требуется определить фрагменты, которые склонны к образованию ошибок секвенирования после них. Также требуется вычислить соответствующую вероятность возникновения такой ошибки. Полученные показатели качества должны повышать эффективность обнаружения мутаций в настоящем геноме, представленным этими чтениями.

\section{Проблемы задачи}

В идеальном случае при полном совпадении референсного генома и настоящего генома, представленном в виде набора его чтений, все несовпадения, встречаемые в чтениях, являются ошибками секвенирования. Но на практике мутации происходят в одной из тысячи позиций. В этом случае не каждое несовпадение является ошибкой, и, наоборот, не каждое совпадение верным. Суммарная погрешность будет зависеть от количества чтений, покрывающих каждую мутацию. Если мутации не учитывать, заметно снизится качество нуклеотидов в таких геномных позициях, что отрицательно повлияет на их обнаружение. Проблема состоит в разделении верных нуклеотидов и ошибочных. Но не всегда наличие большого числа разных нуклеотидов свидетельствует об систематической ошибке, в каком-то из них. Задача усложняется в диплойдном случае, при котором в одной геномной позиции могут быть одновременно два различных верных нуклеотида, так называемый гетерозиготный генотип.


\section{Выводы по главе \protect~\ref{chapter1}}
\label{summary_1}
	Даны общие определения предметной области. Сформулирована основная задача настоящей работы. Описаны проблемы отличия ошибки секвенирования от мутации и способы её решения.